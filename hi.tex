\documentclass[]{article}
\usepackage{lmodern}
\usepackage{amssymb,amsmath}
\usepackage{ifxetex,ifluatex}
\usepackage{fixltx2e} % provides \textsubscript
\ifnum 0\ifxetex 1\fi\ifluatex 1\fi=0 % if pdftex
  \usepackage[T1]{fontenc}
  \usepackage[utf8]{inputenc}
\else % if luatex or xelatex
  \ifxetex
    \usepackage{mathspec}
  \else
    \usepackage{fontspec}
  \fi
  \defaultfontfeatures{Ligatures=TeX,Scale=MatchLowercase}
\fi
% use upquote if available, for straight quotes in verbatim environments
\IfFileExists{upquote.sty}{\usepackage{upquote}}{}
% use microtype if available
\IfFileExists{microtype.sty}{%
\usepackage{microtype}
\UseMicrotypeSet[protrusion]{basicmath} % disable protrusion for tt fonts
}{}
\usepackage[margin=1in]{geometry}
\usepackage{hyperref}
\hypersetup{unicode=true,
            pdftitle={빅데이터 그래픽스 HW1},
            pdfauthor={박효선 1585063},
            pdfborder={0 0 0},
            breaklinks=true}
\urlstyle{same}  % don't use monospace font for urls
\usepackage{color}
\usepackage{fancyvrb}
\newcommand{\VerbBar}{|}
\newcommand{\VERB}{\Verb[commandchars=\\\{\}]}
\DefineVerbatimEnvironment{Highlighting}{Verbatim}{commandchars=\\\{\}}
% Add ',fontsize=\small' for more characters per line
\usepackage{framed}
\definecolor{shadecolor}{RGB}{248,248,248}
\newenvironment{Shaded}{\begin{snugshade}}{\end{snugshade}}
\newcommand{\KeywordTok}[1]{\textcolor[rgb]{0.13,0.29,0.53}{\textbf{#1}}}
\newcommand{\DataTypeTok}[1]{\textcolor[rgb]{0.13,0.29,0.53}{#1}}
\newcommand{\DecValTok}[1]{\textcolor[rgb]{0.00,0.00,0.81}{#1}}
\newcommand{\BaseNTok}[1]{\textcolor[rgb]{0.00,0.00,0.81}{#1}}
\newcommand{\FloatTok}[1]{\textcolor[rgb]{0.00,0.00,0.81}{#1}}
\newcommand{\ConstantTok}[1]{\textcolor[rgb]{0.00,0.00,0.00}{#1}}
\newcommand{\CharTok}[1]{\textcolor[rgb]{0.31,0.60,0.02}{#1}}
\newcommand{\SpecialCharTok}[1]{\textcolor[rgb]{0.00,0.00,0.00}{#1}}
\newcommand{\StringTok}[1]{\textcolor[rgb]{0.31,0.60,0.02}{#1}}
\newcommand{\VerbatimStringTok}[1]{\textcolor[rgb]{0.31,0.60,0.02}{#1}}
\newcommand{\SpecialStringTok}[1]{\textcolor[rgb]{0.31,0.60,0.02}{#1}}
\newcommand{\ImportTok}[1]{#1}
\newcommand{\CommentTok}[1]{\textcolor[rgb]{0.56,0.35,0.01}{\textit{#1}}}
\newcommand{\DocumentationTok}[1]{\textcolor[rgb]{0.56,0.35,0.01}{\textbf{\textit{#1}}}}
\newcommand{\AnnotationTok}[1]{\textcolor[rgb]{0.56,0.35,0.01}{\textbf{\textit{#1}}}}
\newcommand{\CommentVarTok}[1]{\textcolor[rgb]{0.56,0.35,0.01}{\textbf{\textit{#1}}}}
\newcommand{\OtherTok}[1]{\textcolor[rgb]{0.56,0.35,0.01}{#1}}
\newcommand{\FunctionTok}[1]{\textcolor[rgb]{0.00,0.00,0.00}{#1}}
\newcommand{\VariableTok}[1]{\textcolor[rgb]{0.00,0.00,0.00}{#1}}
\newcommand{\ControlFlowTok}[1]{\textcolor[rgb]{0.13,0.29,0.53}{\textbf{#1}}}
\newcommand{\OperatorTok}[1]{\textcolor[rgb]{0.81,0.36,0.00}{\textbf{#1}}}
\newcommand{\BuiltInTok}[1]{#1}
\newcommand{\ExtensionTok}[1]{#1}
\newcommand{\PreprocessorTok}[1]{\textcolor[rgb]{0.56,0.35,0.01}{\textit{#1}}}
\newcommand{\AttributeTok}[1]{\textcolor[rgb]{0.77,0.63,0.00}{#1}}
\newcommand{\RegionMarkerTok}[1]{#1}
\newcommand{\InformationTok}[1]{\textcolor[rgb]{0.56,0.35,0.01}{\textbf{\textit{#1}}}}
\newcommand{\WarningTok}[1]{\textcolor[rgb]{0.56,0.35,0.01}{\textbf{\textit{#1}}}}
\newcommand{\AlertTok}[1]{\textcolor[rgb]{0.94,0.16,0.16}{#1}}
\newcommand{\ErrorTok}[1]{\textcolor[rgb]{0.64,0.00,0.00}{\textbf{#1}}}
\newcommand{\NormalTok}[1]{#1}
\usepackage{graphicx,grffile}
\makeatletter
\def\maxwidth{\ifdim\Gin@nat@width>\linewidth\linewidth\else\Gin@nat@width\fi}
\def\maxheight{\ifdim\Gin@nat@height>\textheight\textheight\else\Gin@nat@height\fi}
\makeatother
% Scale images if necessary, so that they will not overflow the page
% margins by default, and it is still possible to overwrite the defaults
% using explicit options in \includegraphics[width, height, ...]{}
\setkeys{Gin}{width=\maxwidth,height=\maxheight,keepaspectratio}
\IfFileExists{parskip.sty}{%
\usepackage{parskip}
}{% else
\setlength{\parindent}{0pt}
\setlength{\parskip}{6pt plus 2pt minus 1pt}
}
\setlength{\emergencystretch}{3em}  % prevent overfull lines
\providecommand{\tightlist}{%
  \setlength{\itemsep}{0pt}\setlength{\parskip}{0pt}}
\setcounter{secnumdepth}{0}
% Redefines (sub)paragraphs to behave more like sections
\ifx\paragraph\undefined\else
\let\oldparagraph\paragraph
\renewcommand{\paragraph}[1]{\oldparagraph{#1}\mbox{}}
\fi
\ifx\subparagraph\undefined\else
\let\oldsubparagraph\subparagraph
\renewcommand{\subparagraph}[1]{\oldsubparagraph{#1}\mbox{}}
\fi

%%% Use protect on footnotes to avoid problems with footnotes in titles
\let\rmarkdownfootnote\footnote%
\def\footnote{\protect\rmarkdownfootnote}

%%% Change title format to be more compact
\usepackage{titling}

% Create subtitle command for use in maketitle
\newcommand{\subtitle}[1]{
  \posttitle{
    \begin{center}\large#1\end{center}
    }
}

\setlength{\droptitle}{-2em}

  \title{빅데이터 그래픽스 HW1}
    \pretitle{\vspace{\droptitle}\centering\huge}
  \posttitle{\par}
    \author{박효선 1585063}
    \preauthor{\centering\large\emph}
  \postauthor{\par}
      \predate{\centering\large\emph}
  \postdate{\par}
    \date{2018년 9월 16일}


\begin{document}
\maketitle

\subsection{mtcars}\label{mtcars}

\begin{enumerate}
\def\labelenumi{\arabic{enumi})}
\item
  mtcars 자료에 대한 설명으로부터 변수들의 설명을 찾아보시오 {[}, 1{]}
  mpg Miles/(US) gallon {[}, 2{]} cyl Number of cylinders {[}, 3{]} disp
  Displacement (cu.in.) {[}, 4{]} hp Gross horsepower {[}, 5{]} drat
  Rear axle ratio {[}, 6{]} wt Weight (1000 lbs) {[}, 7{]} qsec 1/4 mile
  time {[}, 8{]} vs Engine (0 = V-shaped, 1 = straight) {[}, 9{]} am
  Transmission (0 = automatic, 1 = manual) {[},10{]} gear Number of
  forward gears {[},11{]} carb Number of carburetors
\item
  mtcars 자료의 개수와 변수의 개수를 구하시오. 32 observations 11
  variables
\end{enumerate}

\begin{Shaded}
\begin{Highlighting}[]
\CommentTok{#2) }
\KeywordTok{dim}\NormalTok{(mtcars)}
\end{Highlighting}
\end{Shaded}

\begin{verbatim}
## [1] 32 11
\end{verbatim}

\begin{enumerate}
\def\labelenumi{\arabic{enumi})}
\setcounter{enumi}{2}
\item
  write.csv의 기본 option들을 이용하여 mtcars.csv 파일을 생성하시오.
\item
  생성한 mtcars.csv 파일을 읽어들여 mtcars.read라는 이름의 object를
  만드시오.
\end{enumerate}

\begin{Shaded}
\begin{Highlighting}[]
\CommentTok{#4)}
\NormalTok{mtcars.read<-}\StringTok{ }\KeywordTok{read.csv}\NormalTok{(}\StringTok{"mtcars.csv"}\NormalTok{)}
\end{Highlighting}
\end{Shaded}

\begin{enumerate}
\def\labelenumi{\arabic{enumi})}
\setcounter{enumi}{4}
\tightlist
\item
  mtcars와 mtcars.read의 처음과 끝 6줄씩을 비교하고 차이가 있다면 어떤
  차이가 있는지 기술하시오. mtcars 에서 32 개의 automobiles 이름이
  mtars.read 에서는 x 변수 명으로 들어갔다
\end{enumerate}

\begin{Shaded}
\begin{Highlighting}[]
\CommentTok{#5) }
\KeywordTok{head}\NormalTok{(mtcars); }\KeywordTok{tail}\NormalTok{(mtcars)}
\end{Highlighting}
\end{Shaded}

\begin{verbatim}
##                    mpg cyl disp  hp drat    wt  qsec vs am gear carb
## Mazda RX4         21.0   6  160 110 3.90 2.620 16.46  0  1    4    4
## Mazda RX4 Wag     21.0   6  160 110 3.90 2.875 17.02  0  1    4    4
## Datsun 710        22.8   4  108  93 3.85 2.320 18.61  1  1    4    1
## Hornet 4 Drive    21.4   6  258 110 3.08 3.215 19.44  1  0    3    1
## Hornet Sportabout 18.7   8  360 175 3.15 3.440 17.02  0  0    3    2
## Valiant           18.1   6  225 105 2.76 3.460 20.22  1  0    3    1
\end{verbatim}

\begin{verbatim}
##                 mpg cyl  disp  hp drat    wt qsec vs am gear carb
## Porsche 914-2  26.0   4 120.3  91 4.43 2.140 16.7  0  1    5    2
## Lotus Europa   30.4   4  95.1 113 3.77 1.513 16.9  1  1    5    2
## Ford Pantera L 15.8   8 351.0 264 4.22 3.170 14.5  0  1    5    4
## Ferrari Dino   19.7   6 145.0 175 3.62 2.770 15.5  0  1    5    6
## Maserati Bora  15.0   8 301.0 335 3.54 3.570 14.6  0  1    5    8
## Volvo 142E     21.4   4 121.0 109 4.11 2.780 18.6  1  1    4    2
\end{verbatim}

\begin{Shaded}
\begin{Highlighting}[]
\KeywordTok{head}\NormalTok{(mtcars.read); }\KeywordTok{tail}\NormalTok{(mtcars.read)}
\end{Highlighting}
\end{Shaded}

\begin{verbatim}
##                   X  mpg cyl disp  hp drat    wt  qsec vs am gear carb
## 1         Mazda RX4 21.0   6  160 110 3.90 2.620 16.46  0  1    4    4
## 2     Mazda RX4 Wag 21.0   6  160 110 3.90 2.875 17.02  0  1    4    4
## 3        Datsun 710 22.8   4  108  93 3.85 2.320 18.61  1  1    4    1
## 4    Hornet 4 Drive 21.4   6  258 110 3.08 3.215 19.44  1  0    3    1
## 5 Hornet Sportabout 18.7   8  360 175 3.15 3.440 17.02  0  0    3    2
## 6           Valiant 18.1   6  225 105 2.76 3.460 20.22  1  0    3    1
\end{verbatim}

\begin{verbatim}
##                 X  mpg cyl  disp  hp drat    wt qsec vs am gear carb
## 27  Porsche 914-2 26.0   4 120.3  91 4.43 2.140 16.7  0  1    5    2
## 28   Lotus Europa 30.4   4  95.1 113 3.77 1.513 16.9  1  1    5    2
## 29 Ford Pantera L 15.8   8 351.0 264 4.22 3.170 14.5  0  1    5    4
## 30   Ferrari Dino 19.7   6 145.0 175 3.62 2.770 15.5  0  1    5    6
## 31  Maserati Bora 15.0   8 301.0 335 3.54 3.570 14.6  0  1    5    8
## 32     Volvo 142E 21.4   4 121.0 109 4.11 2.780 18.6  1  1    4    2
\end{verbatim}

\begin{Shaded}
\begin{Highlighting}[]
\KeywordTok{dim}\NormalTok{(mtcars); }\KeywordTok{dim}\NormalTok{(mtcars.read)}
\end{Highlighting}
\end{Shaded}

\begin{verbatim}
## [1] 32 11
\end{verbatim}

\begin{verbatim}
## [1] 32 12
\end{verbatim}

\begin{enumerate}
\def\labelenumi{\arabic{enumi})}
\setcounter{enumi}{5}
\tightlist
\item
  transmission이 automatic인지 아닌지를 TRUE/FALSE로 나타내는 isAuto
  변수를 만들어 mtcars.read에 추가하시오.
\end{enumerate}

\begin{Shaded}
\begin{Highlighting}[]
\CommentTok{#6) }
\NormalTok{mtcars.read}\OperatorTok{$}\NormalTok{isAuto <-}\StringTok{ }\KeywordTok{ifelse}\NormalTok{(mtcars.read}\OperatorTok{$}\NormalTok{am }\OperatorTok{==}\StringTok{ 'automatic'}\NormalTok{, }\OtherTok{TRUE}\NormalTok{, }\OtherTok{FALSE}\NormalTok{)}
\end{Highlighting}
\end{Shaded}

\begin{enumerate}
\def\labelenumi{\arabic{enumi})}
\setcounter{enumi}{6}
\tightlist
\item
  mpg값이 평균보다 크거나 같으면 100, 작으면 0의 값을 갖는 새로운 변수
  mpgR을 만들어 mtcars.read에 추가하시오.
\end{enumerate}

\begin{Shaded}
\begin{Highlighting}[]
\CommentTok{#7)}
\NormalTok{mtcars.read}\OperatorTok{$}\NormalTok{mpgR <-}\StringTok{ }\KeywordTok{ifelse}\NormalTok{(mtcars.read}\OperatorTok{$}\NormalTok{mpg }\OperatorTok{>=}\StringTok{ }\KeywordTok{mean}\NormalTok{(mtcars.read}\OperatorTok{$}\NormalTok{mpg), }\DecValTok{100}\NormalTok{, }\DecValTok{0}\NormalTok{)}
\end{Highlighting}
\end{Shaded}

\begin{enumerate}
\def\labelenumi{\arabic{enumi})}
\setcounter{enumi}{7}
\tightlist
\item
  mtcars 자료의 모든 변수들의 특성을 살펴보고 특성에 맞게 정리하시오.
  (정수형 변수인 경우는 범주형변수로 간주하여 정리할 것)
\end{enumerate}

\begin{itemize}
\tightlist
\item
  연속형 변수 mpg, drat, wt, qsec, mpgR
\item
  범주형 변수 x, cyl, hp, vs, am, gear, carb
\item
  논리형 변수 isAuto
\end{itemize}

\begin{Shaded}
\begin{Highlighting}[]
\CommentTok{#8)}
\KeywordTok{str}\NormalTok{(mtcars.read)}
\end{Highlighting}
\end{Shaded}

\begin{verbatim}
## 'data.frame':    32 obs. of  14 variables:
##  $ X     : Factor w/ 32 levels "AMC Javelin",..: 18 19 5 13 14 31 7 21 20 22 ...
##  $ mpg   : num  21 21 22.8 21.4 18.7 18.1 14.3 24.4 22.8 19.2 ...
##  $ cyl   : int  6 6 4 6 8 6 8 4 4 6 ...
##  $ disp  : num  160 160 108 258 360 ...
##  $ hp    : int  110 110 93 110 175 105 245 62 95 123 ...
##  $ drat  : num  3.9 3.9 3.85 3.08 3.15 2.76 3.21 3.69 3.92 3.92 ...
##  $ wt    : num  2.62 2.88 2.32 3.21 3.44 ...
##  $ qsec  : num  16.5 17 18.6 19.4 17 ...
##  $ vs    : int  0 0 1 1 0 1 0 1 1 1 ...
##  $ am    : int  1 1 1 0 0 0 0 0 0 0 ...
##  $ gear  : int  4 4 4 3 3 3 3 4 4 4 ...
##  $ carb  : int  4 4 1 1 2 1 4 2 2 4 ...
##  $ isAuto: logi  FALSE FALSE FALSE FALSE FALSE FALSE ...
##  $ mpgR  : num  100 100 100 100 0 0 0 100 100 0 ...
\end{verbatim}

\begin{Shaded}
\begin{Highlighting}[]
\KeywordTok{summary}\NormalTok{(mtcars.read)}
\end{Highlighting}
\end{Shaded}

\begin{verbatim}
##                   X           mpg             cyl             disp      
##  AMC Javelin       : 1   Min.   :10.40   Min.   :4.000   Min.   : 71.1  
##  Cadillac Fleetwood: 1   1st Qu.:15.43   1st Qu.:4.000   1st Qu.:120.8  
##  Camaro Z28        : 1   Median :19.20   Median :6.000   Median :196.3  
##  Chrysler Imperial : 1   Mean   :20.09   Mean   :6.188   Mean   :230.7  
##  Datsun 710        : 1   3rd Qu.:22.80   3rd Qu.:8.000   3rd Qu.:326.0  
##  Dodge Challenger  : 1   Max.   :33.90   Max.   :8.000   Max.   :472.0  
##  (Other)           :26                                                  
##        hp             drat             wt             qsec      
##  Min.   : 52.0   Min.   :2.760   Min.   :1.513   Min.   :14.50  
##  1st Qu.: 96.5   1st Qu.:3.080   1st Qu.:2.581   1st Qu.:16.89  
##  Median :123.0   Median :3.695   Median :3.325   Median :17.71  
##  Mean   :146.7   Mean   :3.597   Mean   :3.217   Mean   :17.85  
##  3rd Qu.:180.0   3rd Qu.:3.920   3rd Qu.:3.610   3rd Qu.:18.90  
##  Max.   :335.0   Max.   :4.930   Max.   :5.424   Max.   :22.90  
##                                                                 
##        vs               am              gear            carb      
##  Min.   :0.0000   Min.   :0.0000   Min.   :3.000   Min.   :1.000  
##  1st Qu.:0.0000   1st Qu.:0.0000   1st Qu.:3.000   1st Qu.:2.000  
##  Median :0.0000   Median :0.0000   Median :4.000   Median :2.000  
##  Mean   :0.4375   Mean   :0.4062   Mean   :3.688   Mean   :2.812  
##  3rd Qu.:1.0000   3rd Qu.:1.0000   3rd Qu.:4.000   3rd Qu.:4.000  
##  Max.   :1.0000   Max.   :1.0000   Max.   :5.000   Max.   :8.000  
##                                                                   
##    isAuto             mpgR       
##  Mode :logical   Min.   :  0.00  
##  FALSE:32        1st Qu.:  0.00  
##                  Median :  0.00  
##                  Mean   : 43.75  
##                  3rd Qu.:100.00  
##                  Max.   :100.00  
## 
\end{verbatim}


\end{document}
